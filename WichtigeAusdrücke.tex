\documentclass[10pt,a4paper]{article}
\usepackage[utf8]{inputenc}
\usepackage[german]{babel}
\usepackage[T1]{fontenc}
\usepackage{amsmath}
\usepackage{amsfonts}
\usepackage{amssymb}
\usepackage{graphicx}
\usepackage{lmodern}
\usepackage{fourier}
\usepackage{listings}
\usepackage{xcolor}
\usepackage[left=2cm,right=2cm,top=2cm,bottom=2cm]{geometry}

\lstset{
    language=Java,
    basicstyle=\ttfamily\small,
    keywordstyle=\color{blue}\bfseries,
    stringstyle=\color{red},
    commentstyle=\color{green!50!black},
    numbers=left,
    numberstyle=\tiny\color{gray},
    stepnumber=1,
    breaklines=true,
    tabsize=4,
    showspaces=false,
    showstringspaces=false
}

\begin{document}
\section{Eigenschaften von Algos}

\subsection{Terminierung}
Algo wird nach endlich vielen Schritten beendet

\subsection{Determiniert}
Für selben zulässigen Eingabewert stets dasselbe Ergebnis

\subsection{Deterministisch}
Für Eingabedaten Reihenfolge aller auszuführenden Schritte eindeutig bestimmt

\subsection{Partiell Korrekt}
es läuft einfach

\subsection{Total Korrekt}
Partiell korrekt und muss terminieren

\section{Datenrepäsentation}
\begin{itemize}
\item Bitwerte: 0, 1
\item Byte = 8 Bits = 1 Octet, Byte nimmt $2^8=256$
\end{itemize}

\section{Datentypen}
\begin{enumerate}
\item int: 32 Bit
\item float: 32 Bit
\item double: 64 Bit
\item char: 16 Bit
\end{enumerate}

\section{Laufzeitspeicher}
\begin{itemize}
\item Stack für Variablen und Parameter in Methoden
\item Heap für Werte von Referenzdaten
\end{itemize}

\section{Sortieralgo}
\begin{itemize}
\item InsertionSort: Sortieren wie beim Kartenstapel durch Mensch
\item Bubblesort: benachbarte Elemente vergleichen
\end{itemize}

\section{O-Notation}
\begin{itemize}
\item $\mathcal{O}(1)$: i<15
\item $\mathcal{O}(n)$: i<n
\item $\mathcal{O}(n^2)$: Schleife in einer Schleife, i<n und j<n
\end{itemize}

\section{Lambda-Ausdruck}
Parameter -> Anweisung

\section{Exceptions}
\begin{lstlisting}
try {
	irgendne Methode
} catch (<ExceptionType> e){
	e.print...
} finally {
	fuehre danach aus
}
\end{lstlisting}

\section{Assertions}
Selber mit assert an beliebiger Stelle prüfen, ob Code richtig läuft, wie vorgestellt
Aktivierung der Überprüfung in Konsole/Terminal
\begin{lstlisting}
java -ea ClassFile
\end{lstlisting}

\end{document}